\documentclass[a4paper]{article}

% Includes packages relevant to Senior Lab

% character set specifications
\usepackage[english]{babel}
\usepackage[utf8]{inputenc}

% extra unicode characters
\DeclareUnicodeCharacter{3BC}{\(\mu\)}
\DeclareUnicodeCharacter{3C1}{\(\rho\)}
\DeclareUnicodeCharacter{2080}{\(_0\)}
\DeclareUnicodeCharacter{2081}{\(_1\)}
\DeclareUnicodeCharacter{2082}{\(_2\)}

% SI Units
\usepackage{siunitx}

% extra SI units
\DeclareSIUnit\gauss{G}

% enable scientific notation
\sisetup{scientific-notation = engineering, exponent-to-prefix}

% draw pretty lines
\usepackage{tikz}
\usetikzlibrary{datavisualization}
\usepackage{circuitikz}

% manual tabbing
\setlength{\parindent}{0pt}
\def\qq{\qquad}

% include graphics
\usepackage{graphicx}

% increased control over figure placement
\usepackage{float}

% box answers
\usepackage{tcolorbox}

% enable multiple section levels
\usepackage{titlesec}

% define `\subsubsubsection` command
\titleclass{\subsubsubsection}{straight}[\subsection]
\newcounter{subsubsubsection}[subsubsection]
\renewcommand\thesubsubsubsection{\thesubsubsection.\arabic{subsubsubsection}}
\titleformat{\subsubsubsection}
        {\normalfont\normalsize\bfseries}{\thesubsubsubsection}{1em}{}
\titlespacing*{\subsubsubsection}
{0pt}{3.25ex plus 1ex minus .2ex}{1.5ex plus .2ex}
\setcounter{secnumdepth}{4}

% get align environment (among other things)
\usepackage{amsmath}

% bold in math mode
\usepackage{bm}

% get \mathbb (among other things)
\usepackage{amssymb}

\usepackage{array}

% plotting
\usepackage{pgfplots}

% enable external references
\usepackage{hyperref}

% include code
\usepackage{minted}
\setminted{linenos, frame=lines, texcomments}


\usepackage{caption}
\renewcommand{\thetable}{\arabic{section}.\arabic{table}}
\newcommand\T{\rule{0pt}{2.6ex}}       % Top strut
\newcommand\B{\rule[-1.2ex]{0pt}{0pt}} % Bottom strut

\title{PHY 4210-01 Senior Lab \\Lab P1: Planck's Constant\\Lab P6: Blackbody Radiation}

\author{Sarah Arends \\
        Jacquelyne Miksanek \\
        Ryan Wojtyla \\ \\
        Instructor: Augustus Azelis}

\date{\today}

\begin{document}
\maketitle

\begin{abstract}
%physics of experiment
%apparatus used
%what was measured
%Results
\qq 
\end{abstract}

\newpage

\tableofcontents

\newpage

P1 Planck's Constant Experiment

\newpage

\section{Objective of the Experiment}
%A brief statement on the main purpose of the experiment
\qq When demonstrating the photoelectric effect, the stopping
potential for a multitude of wavelengths is measured. The resultant
data is used to determine a value for Planck's constant through
experimentation with a photocell's cathode and a mercury lamp.

\section{Theory of the Experiment}
\qq A photon is a quantized bundle of energy that carries momentum,
but has no mass. These photons make up a general term of light. When
light comes into contact with metals, the photon energizes the atom of
the metals. If sufficient enough this energy will cause the unbound
electrons in the metal to absorb the energy and leave the metal. This
is known as the photoelectric effect. An anode can then collect the
electrons and an electrical current is the result. It is important to
note that increasing the intensity of the light will not increase the
kinetic energy of the electrons. Applying a voltage between and anode
and a canthode will exert a force toward the anode. When the voltage
is increased to the stopping potential the quantity of energy is then
equal to the kinetic energy of the electron. If the the law of
conservation of energy holds true then the kinetic energy of the
electron should be equal to the energy given to the electron by the
photon minus the energy that is transferred when the electron leaves
the metal. When the voltage exceeds the stopping potential voltage,
the photocurrent reverses and a negative current is registered on the
ammeter. The kinetic energy of light is equated to be $ E = hf$, where
$E$ is the energy, $h$ is the Planck constant, and $f$ is the
frequency. The work function of the metal is given by $W$ in the
following equation: $ W = h \times f - e \times V_0$. $V_0$ is the
stopping potential, and $e \times V_0$ is the kinetic energy. The work
function describes the quantity of energy that is necessary for a
photon to release an electron from the metal.

\section{Equipment Utilized}
%List principal pieces of apparatus used by manufacturer, model and
%serial number. When it may be important, list principal
%specifications of certain pieces of equipment (e.g. the focal length
%of an optical system, etc.)

% Description of set-up in prose
\qq A DC power supply is connected in parallel to a voltmeter. The
voltmeter will monitor the voltage of the setup and will be used to
take note of the stopping potential. A Keithley Picoammeter, and
photocell are connected in parallel to the DC power supply. The
Picoammeter will read the resultant current. A Mercury Lamp will be
placed a distance apart from the photocell, with the circular bulb
opening in allignment with the photocell's receptor window. When the
spectral filters are applied, they will be slid into the slotted grove
directly infront of the circular bulb opening on the Mercury Lamp.

% List of specs
\begin{itemize}
\item Phototube \\
\item Mercury Lamp \\
\item Spectral Filters \\
\item Daedelon Corporation Photocell \\
\item DC Power Supply \\
\item Keithley Picoammeter \\
\item Voltmeter \\
\end{itemize}

%Labeled sketch of the experimental setup
\begin{figure}[H]
\centering
% uncomment the line below to add image
%\includegraphics[width=1\textwidth]{diagram.png}
\captionof{figure}{Schematic for circuit diagram for the Photoelectric
  Effect}
\label{Photoelectric Effect Circuit Diagram}
\end{figure}

\section{Procedure}

% Describe the main steps in the experimental procedures. Be sure to include any
% precautions. Sufficient details should be given such that another student can
% follow and do the experiment.

\qq The experiment begins by turning on the Mercury Lamp, a constant
level of brightness is acheived by letting the Mercury Lamp warm-up
for conservatively an hour. In the interest of time the Mercury Lamp
was warmed-up for approximately twenty minutes. Remember to have the
Mercury Lamp directed toward a wall for safety. Ensure that the
photocell's annode and cathode is connected directly to the
Picoammeter, use caution when handling the wires on the photocell as
they are fragile and fray easily. Do not allow the applied voltage to
exceed 40 volts and the current to exceed 20 milliamps. The
Picoammeter should be configured to readout the current in units of
microamps. To measure the ambiant background, point the photocell's
window toward the ceiling. If the experiment is to be completed with
the room lights off, then the background should be measured in the
same setting. The photocurrent of the ambiant background can then be
measured to an uncertianty of $\pm$ 1 nanoamp. The next measurement to
be taken is the unfiltered light from the Mercury Lamp. This time the
window of the photocell should be pointed toward the circular bulb
opening on the lamp, ensure that the opening is alligned with the
window. The resultant photocurrent is then recorded. The following
steps are used to measure the stopping potential for a spectra of
differing wavelengths using spectral filters. The wires to the
Picoammeter are disconnected and a DC power supply is added to the
circuit, keeping the power supply switched off. The voltage control
knob should be set to the lowest possible setting, ideally zero. The
current limit swicth should be limited to 0.5 amps, and then the
current is adjusted to a moderate position within it. Turn the knob on
the photocell to its lowest setting. This is the photocell's built-in
potentiometer. The circuit can then be assmbled as depicted in
\ref{photo_circuit}. Before the power supply is switched on, the
entire circuit and settings should be double-checked to ensure the
safety of the equipment. If the circuit needs to be altered it is
important to place the Picoammeter into zero-check mode (denoted by
ZCH), this is a safe mode for the ammeter to ensure it is not
overloaded. Slowly turn the voltage on the DC power supply up to 6
volts. The stopping potential should not be increasing here, this can
be monitored on the voltmeter. Slowly the resistance of the
potentiometer should be increased, and the stopping potential,
displayed on the voltmeter, should increase negatively between 0 and
-2 volts. It is important to note that while this value is becoming
more negative the current should be slowly increasing from a negative
number to zero. The stooping potential is reached when the current is
equal to zero. Record the measurements during this process, and plot
the potenital versus the current. This is repeated for all of the
varying filters that accompanied the Mercury Lamp. A graph of the
potential versus the light frequency is then created. The slope of
this graph is the experimental Planck's constant value.

\subsection{Procedural Modifications}

\qq It is important to note that the potentiometer on the photocell
had a range of 0 to -1.5 volts, and thus the whole range was not
tested as it could not be. This differs from the lab manual. It is
also important to note that the stopping potential was not deterimed
for all filters as some had a stopping potential greater than the
absolute value of the -1.5 volt maximum of the photocell.

\subsection{Safety Tips}

\qq Please note that the Mercury Lamp emitts ultra-violet light and
looking directly at the bulb can cause cataracts in the eyes. Always
ensure that the Mercury Lamp is facing a wall and is not shining on
anyone in the laboratory.


\section{Data Analysis}

%Graphs, figures, and tables with captions
%Results with error analysis
%Calculate discrepancies from theory

\qq While some filters appeared to be similar in color they were not
the same filters and every filter was indeed labeled differently. The
darker filters reduced the intensity of the light that interacts with
the photocell and does affect the stopping potential for these
filters. These darker filters have a stopping potential of a higher
absolute value, then the filters that were lighter in appearence. 

%%%%%%% NEEDS GRAPHS I_V AND I_F! 
%%%%%%% NEEDS PLANCK'S CONSTANT! 
%%%%%%% NEEDS PROPAGATED ERROR! 

\section{Results}

%%%%%% NEEDS SIGMA STUFF!

\subsection{Source of Error}

\qq 

\section{Conclusion}
%Brief summary, discussion of theory

\newpage

P6: Blackbody Radiation

\newpage

\section{Objective of the Experiment}
%A brief statement on the main purpose of the experiment
\qq 

\section{Theory of the Experiment}


\section{Equipment Utilized}
%List principal pieces of apparatus used by manufacturer, model and
%serial number. When it may be important, list principal
%specifications of certain pieces of equipment (e.g. the focal length
%of an optical system, etc.)

% Description of set-up in prose
\qq 

% List of specs
\begin{itemize}
\item itemName \\
\item itemName \\
\end{itemize}

%Labeled sketch of the experimental setup
\begin{figure}[H]
\centering
% uncomment the line below to add image
%\includegraphics[width=1\textwidth]{diagram.png}
\captionof{figure}{Schematic for circuit diagram }
\label{Diagram}
\end{figure}

\section{Procedure}

% Describe the main steps in the experimental procedures. Be sure to include any
% precautions. Sufficient details should be given such that another student can
% follow and do the experiment.

\subsection{Procedural Modifications}
\qq 

\section{Data Analysis}

%Graphs, figures, and tables with captions
%Results with error analysis
%Calculate discrepancies from theory

\section{Results}

\qq For each of the four voltages, the relative intensity of light incident upon
the broad spectrum light sensor was plotted against the wavelength of the light
being measured by the sensor.

\qq First, \SI{3}{\volt} and \SI{0.388}{\ampere} were applied to the
bulb. Although no clear pattern can be readily discerned from the plot, Figure
\ref{gph:3volt}, at lower wavelengths, a clear peak of intensity occurs at
\( 939 \pm 1 \si{\nano\meter} \). This value is within the range of the
theoretical peak wavelength of \( 1734 \pm 1292 \si{\nano\meter} \). The
discrepancy between the theoretical and experimental values of the peak
wavelength is calculated with
\( \Delta_{\lambda} = | \lambda_t - \lambda_e | \), where \( d \) is the
discrepancy, \( \lambda_t = 939 \pm 1 \si{\nano\meter} \) is the theoretical
value of the peak wavelength, and
\( \lambda_e = 1734 \pm 1292 \si{\nano\meter} \) is the experimental value of
the peak wavelength.

\begin{align*}
  \Delta_{\lambda} =& \left| (939) - (1734) \right| \\
  \Delta_{\lambda} =& 795 \si{\nano\meter} \\
\end{align*}

Since the standard deviation of the wavelengths at \SI{3}{\volt} is \(
\sigma_{\lambda} = \SI{306}{\nano\meter} \), the experimental value of \(
\lambda \) at \SI{3}{\volt} is:

\begin{equation*}
  \frac{\Delta_{\lambda}
\end{equation*}

\subsection{Source of Error}

\section{Conclusion}
%Brief summary, discussion of theory

\section{Appendices}

\end{document}
