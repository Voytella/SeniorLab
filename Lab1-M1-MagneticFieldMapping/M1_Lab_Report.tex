\documentclass[a4paper]{article}

\usepackage[english]{babel}
\usepackage[utf8]{inputenc}
\usepackage{amsmath}
\usepackage{graphicx}
\usepackage[colorinlistoftodos]{todonotes}
\usepackage{caption}
\usepackage{pgfplots}
\pgfplotsset{width=10cm,compat=1.9}
\renewcommand{\thetable}{\arabic{section}.\arabic{table}}
\newcommand\T{\rule{0pt}{2.6ex}}       % Top strut
\newcommand\B{\rule[-1.2ex]{0pt}{0pt}} % Bottom strut

\title{PHY 4210-01 Senior Lab \\Lab M-1: Magnetic Field Mapping}

\author{Sarah Arends \\ 
        Jacquelyne Miksanek \\
        Ryan Wojtyla \\ \\
        Instructor: Jerry Collins II}

\date{February 7, 2019}

\begin{document}
\maketitle 

\begin{abstract}
%physics of experiment
%apparatus used
%what was measured
%Results
In this experiment the magnetic field inside a Helmholtz coil was measured and compared to theoretical calculations determined from the Smythe derivation of the Biot-Sarvat Law for a plane displaced from the central axis, with coordinates z, $\rho$, and $\phi$.When determining the magnetic field inside a Helmholtz coil, a Hall probe is used to obtain the magnitude of the magnetic field at varying positions inside the coil. Theoretically the axial component of the magnetic field that is produced inside the Helmholtz coil is, to some extent, of uniform magnitude.    
\end{abstract}

\newpage

\tableofcontents

\newpage

\section{Objective of the Experiment}
%A brief statement on the main purpose of the experiment
During this lab, the number of turns of wire inside a Helmholtz coil
was determined for use in theoretical calculations. Then a
3-dimensional and 2-dimensional mapping of the magnetic field inside
the Helmholtz coil was created in order to investigate the presence of
a uniform field, running along its axial direction.

\section{Theory of the Experiment}
%A short presentation of the concepts and formulas related to the experiment.

% How a helmholtz coil produces a uniform field
Recall for a straight current-carrying wire, circular magnetic field
lines are generated around the wire in accordance with the curling
right-hand rule. The Helmholtz coil contains two regions of circularly
wound wires. Due to the the circular symmetry, all components of each
infinitesimal segment of the wire will cancel \textit{except} for that
in the axial direction. In summary, a circular current produces a
linear magnetic field.

% off-axis field point
The field point of the system has before been typically placed along
the axis of the direction of the magnetic field, we will call this the
z-direction. This was due to the ease of solving the Biot-Savart Law
under these simple conditions, as the direction and strength of the
magnetic field will follow along the z-axis of the system, which is
where the field point is placed. When this is applied to the co-axial
coils of the Helmholtz apparatus the evaluation of the Biot-Savart Law
becomes too trivial. One then chooses the field point to be placed off
of the z-axis as more information about the magnetic field of the
coils can be determined. This is the more general scenario and thus
more complex. The off axis form can be used for any point that is off
of the z-axis, while the on axis is a specific and simplified form of
the general case. The general form is best represented by Smythe's
derivation of the Biot-Savart Law.

\begin{align*} 
B_z = \frac{\mu_0IN}{2\pi}
\Big[&
    \frac{1}{\sqrt{(a+\rho)^2 + (a-z)^2}}
    \big[
        K_1 + \left(\frac{a^2 -\rho^2 - (a - z)^2}{(a-\rho)^2 + (a - z)^2} \right) E_1 
    \big]\\
    & + \frac{1}{\sqrt{(a + \rho)^2 + z^2}}
    \big[
        K_2 + \big(\frac{a^2 - \rho^2 - z^2}{(a - \rho)^2 + z^2}\big) E_2 
    \big] 
\Big]
\end{align*}

\section{Equipment Utilized}

%List principal pieces of apparatus used by manufacturer, model and serial number. When it may be important, list principal specifications of certain pieces of equipment (e.g. the focal length of an optical system, etc.)

Helmholtz coil \\
Gauss meter \\
Hall probe \\
Meterstick \\
Ruler \\
Dipmeter \\
Powersource \\
Magnaprobe \\
Multimeter\\

\subsection{The Helmholtz coil}

% circuit of the helmholtz coil
The Helmholtz coil consists of two concentric sets of coils, each with
the same radius and separated by a distance equal to their
radius. This configuration allows the contribution of each set of
coils to produce a uniform field in the center of the coils. The
current in each set of coils must be oriented in a particular
direction so that their contributions constructively interfere. The
circuit is shown in figure \ref{helmholtz_circuit}.

% circuit diagram of coils
\begin{figure}[h]
\centering
% uncomment the line below to add image
%\includegraphics[width=0.5\textwidth]{IMAGE_NAME.png}
\captionof{figure}{Flow of current through the Helmholtz coil,
  oriented such that the produced fields are constructive.}
\label{helmholtz_circuit}
\end{figure}

\subsection{The Hall Effect Probe}

%How hall effect probe works
A DC Gaussmeter (AlphaLab Model GM-1-HS) was connected to a Hall
Effect Probe in order to measure the field strength inside the
Helmholtz coil. The Hall Effect Probe contains a semiconductor
junction that, when exposed to a magnetic field, produces a voltage
proportional to the field strength.

\subsection{Position Controls}

%Position Controls
The position of the Hall Effect Probe can be modified in the $\rho$
direction by sliding the ruler bar through the acrylic cube shown in
figure \ref{helmholtz_diagram}. The position can be modified in the
$\phi$ direction by rotation the ruler bar about the central
pole. However, for the sake of this experiment, this did not have to
be modified because measurements were taken in a single $\rho , z$
plane. The $z$ coordinate was modified by sliding the acrylic cube and
ruler bar up and down the central pole.

%Labeled sketch of the experimental setup
\begin{figure}[h]
\centering
% uncomment the line below to add image
\includegraphics[width=0.5\textwidth]{helmholtz_diagram.png}
\captionof{figure}{Two concentric Helmholtz coils separated by a
  distance equal to their radius. Rotating pole and sliding ruler
  allow for modification of the probe's position.}
\label{helmholtz_diagram}
\end{figure}

\section{Procedure}

%Describe the main steps in the experimental procedures. Be sure to include any precautions. Sufficient details should be given such that another student can follow and do the experiment.
Note that, per suggestion of the laboratory manual, the procedural
steps of this experiment have been omitted. The discussion section
provides sufficient detail on what actions were taken.

\subsection{Measuring the External Field}
% How the external field was measured
The Helmholtz coil is oriented such that the Earth's magnetic field is
parallel to the z-axis of the coils. This allows us to produce an
applied magnetic field that is exactly anti-parallel to the Earth's
field. From there, we can compute the applied field by subtracting the
Earth's field from the total resultant field.

Note that there was an apparent offset in the Gaussmeter reading, as
the 0.36G measurement for Earth's field was consistently higher than
the expected value for Earth's field of 0.24G. However, if there truly
exists such an offset in the measurement device, it would appear in
both the measurement of Earth's field and in the measurement of the
total field inside the Helmholtz coil. Subtracting these two to obtain
the strength of the applied field would cancel any contribution from
such an offset.

\subsection{Procedural Modifications}

% Misaligned axis
Upon initial inspection of the equipment, it appeared the center pole
running along the z axis of the Helmholtz coil was misaligned. In
order to mitigate this error and ensure that coordinates were modified
independently, a chord was used to realign the pole as closely as
possible to the true z axis. However, since this alignment was not
quantified, it is possible that there the pole is misaligned to some
degree. This would result in a systematic error intrinsic to the
experimental set-up. If the pole deviates from the z-axis, the
experimentally recorded z-values are underestimated, causing the
experimental field strengths to trend lower than the theoretical field
strengths.

% Day-to-day current offset
The majority of field strength measurements collected for the
3-dimensional mapping were taken on the same day of
experimentation. After resuming this data collection on the next day,
the values appeared to be systematically higher. Possible causes of
this offset were investigated. Before taking measurements and
intermittently during the data collection, the hall effect probe was
zeroed and observed with the power supply off in order to ensure a
consistent reading of the Earth's magnetic field. The reference
measurement taken at the start of this lab session was similar to
those taken during the previous session (zeroed field measurements
were between 0.36G and 0.4G on both days), so a discrepancy in the
Earth's field strength measurement was eliminated as the source of
this error. Note that any small variation in the Earth's field
measurement could be due to misalignment of the probe (a systematic
error in measurement that would under-report the field strength) or
simply a random error in measurement due to the limited performance of
the probe.

An ammeter was also used to ensure a 2A current was consistently
applied on both days of data collection, thus a change in the applied
current was eliminated as a source of error. Because the source of
this error was ultimately not determined and eliminated, the effect
had to be compensated for with a procedural modification. In order to
recreate the data points from the previous lab session, the current
from the power supply was modified until the field strength matched
previous measurements in several locations. This ultimately required
lowering the applied current from 2000mA to 1790mA.

Upon further investigation, it appeared the current from the power
supply was unstable, as it would decrease and increase every few
minutes. This produced a source of random intrinsic error, which was
mitigated by fine tuning the current value before each measurement
after the issue was discovered.

\subsection{Additional Sources of Error}

% Proximity to power supply
Because the experimental set-up was restricted to a small area, the
contribution from the field produced by the power supply may be
non-negligible. From the perspective of the experimenter, the power
supply sits behind and to the right of the Helmholtz coil. Therefore,
by the curling right hand rule, this would produce an upward magnetic
field on the side of the wire nearest the Helmholtz coil. This would
produce a systematic intrinsic error that causes the external field
measurements to be overestimated. Similarly, the power supply itself
may be producing a small field that could also contribute a systematic
error, although the exact effect could not be determined without
knowing the orientation of such a field.

\section{Data Analysis}
%Graphs, figures, and tables with captions
%Results with error analysis
%Calculate discrepancies from theory


\subsection{Calculating Supply Voltage}

Using a multimeter, the resistance of a set of coils was measured to
be $3.4 \Omega$. In order to determine the necessary voltage to send
3A of current through the coils, we make a simply calculation using
Ohm's law.

% Calculating voltage
\begin{align*}
V &= IR \\
  &= (3A)(3.4 \Omega) \\
  &= 10.2V \\
\end{align*}

% Calculating number of turns
\subsection{Determining the Number of Turns in a Coil}

Further calculations will require knowing the number of turns of wire
in each set of coils. MORE TEXT TO COME HERE
\subsection{The Resultant Plots}
%Theoretical calculations of axial field strength

% 3D plot (theoretical)
\begin{figure}
\centering
\includegraphics[width=0.5\textwidth]{3DPlotTheoretical.png}
\captionof {figure}{The theoiretical three dimensional plot calculated
  by using Symthe's deviation of the Biot-Savart Law.}
\label{Theoretical 3-D plot}
\end{figure} 
% 2D plot (theoretical)
\begin{figure}
\centering
\includegraphics[width=0.5\textwidth]{2DPlotTheoretical.png}
\captionof {figure}{The theoretical two dimensional used for
  comparison to the data obtained during the experimentation.}
\label{Theoretical 2-D plot}
\end{figure}
% 3D plot (exp)
\begin{figure}
\centering
\includegraphics[width=0.5\textwidth]{3DPlotExperimental.png}
\captionof {{figure}{The three dimensional plot with the data obtained
    during the experimentation. Where the magnetic field of the
    Helmhotz coil is mapped out to provide a visual of how the
    magnetic field is shaped and its strengths and differing areas. }
\label{Experimental 3-D plot}
\end{figure}
% 2D plot (exp)
\begin{figure}
\centering \includegraphics[width=0.5\textwidth]{2_Dmapping_M_1.png}
\captionof {figure}{The two dimensional plot witht the data obtained
  during the experimentation. Where each line corresponds to a
  differing value along the z-axis of the Helmholtz coil as a function
  of the magnetic field strength in $\mu$ Tesla and the radial
  distance from the z-axis in cm.}
\label{Experimental 2-D plot}
\end{figure}

% Calculating discrepancies

% 3D plot discrepancies 

% 2D plot discrepancies

\section{Results}
%Discuss results and uncertainties
%Compare results with theory
%Approximations to theory
\subsection{Comparing the Theoretical Plots to the Experimental Plots}
The experimental three dimensional plot does follow the theoretical
plot in the sense that the data in the mid region should be clustered
closer together than the edge regions. This is due to the uniformity
of the magnetic field in this region. Whereas the regions closest to
the coils are exposed to a greater magnetic field due to their closer
proximity to the coils themselves. The area nearest to the power
source also draws interference from the power source and the magnetic
field and thus the magnetic field is skewed to be stronger. The
experimental three dimensional plot is seemingly flipped 180 degrees
due to the fact the experiemental magnetic field was measured to be
negative while the theoretical magnetic field was calculated to be
positive.

The experimental two dimensional plot does follow the theoretical plot
in the sense that the data in the mid region is clustered close
together. The derivatives of this area of the graph approach a
plateu, this is recognized to be the center region of the coils where
the uniformity of the magnetic field is most prominent. The measured
magnetic field was determined to be much weaker than the theoretical
magnetic field. Our measured field was approximately an order of
magnitude weaker than the theoretical. The magnetic field was once
again measured to be of negative value while the theoretical magnetic
field was determined to be of positive value.

% Determine span of uniform region with 1percent margin and 5percent margin
\subsection{Determining the Span of the Uniform Region with Margins}
% Compare B field in different direction, can we say field is axial
\subsection{Comparing the directions of the Magnetic Field}
When measuring at a probe height of a/2 (16cm), where 'a' is the
separation distance between the coils, the strength of the magnetic
field in the 'z' direction was measured to be -3.13 Gauss. When
measuring t he magnetic field in the 'z' direction at a probe height
of 5cm, the magnetic field strength was measure d to be -3.28
Gauss. These results follow with the theory as it is expected that the
magnetic field is p ropagated in the 'z' direction. The measured
magnetic field strength for the $\rho$ direction was -0.46 and -0.05
Gauss for a probe height of 16cm and 5cm respectfully. The measured
magnetic field strength fo r the $\phi$ direction was -0.51 and -0.31
Gauss for a probe height of 16cm and 5cm respectfully. For a probe
height of 16cm the percentage for the magnitude of the magnetic field
that is measured to be in th e $\rho$ direction is 14$\%$ while the
percentage for the magnitude of the magnetic field that is measur ed
to be in the $\phi$ direction is 16$\%$. For a probe height of 5cm the
percentage for the magnitude o f the magnetic field that is measured
to be in the $\rho$ direction is 1$\%$ while the percentage for th e
magnitude of the magnetic field that is measured to be in the $\phi$
direction is 9$\%$.The magnetic f ield produced by the Helmholtz coils
should be directed along the 'z' axis. These small measuredvalues f
ollow the aforementioned theoryand we can determine that the magnetic
field produced by the Helmholtz c oil is indeed axial. Furthermore, we
can determine thatthe magnetic field is axial along the 'z' direct
ion.

\section{Conclusion}
%Brief summary, discussion of results and theory

\end{document}
