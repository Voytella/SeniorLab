\documentclass[a4paper]{article}

\usepackage[english]{babel}
\usepackage[utf8]{inputenc}
\usepackage{amsmath}
\usepackage{graphicx}
\usepackage[colorinlistoftodos]{todonotes}
\usepackage{caption}
\usepackage{pgfplots}
\pgfplotsset{width=10cm,compat=1.9}
\renewcommand{\thetable}{\arabic{section}.\arabic{table}}
\newcommand\T{\rule{0pt}{2.6ex}}       % Top strut
\newcommand\B{\rule[-1.2ex]{0pt}{0pt}} % Bottom strut

\title{PHY 4210-01 Senior Lab \\Lab M-1: Magnetic Field Mapping}

\author{Sarah Arends \\ 
        Jacquelyne Miksanek \\
        Ryan Wojtyla \\ \\
        Instructor: Jerry Collins II}

\date{February 7, 2019}

\begin{document}
\maketitle 

\begin{abstract}
%physics of experiment
%apparatus used
%what was measured
In this experiment the magnetic field inside a Helmholtz coil was measured and compared to theoretical calculations determined from the Smythe derivation of the Biot-Sarvat Law for a plane displaced from the central axis, with coordinates z, $\rho$, and $\phi$. 
% second sentence of results still needed!!!
\end{abstract}

\newpage

\tableofcontents

\newpage

\section{Objective of the Experiment}
%A brief statement on the main purpose of the experiment
During this lab, the number of turns insde a Helmholtz coil will be determined to use for future derivations and calculations. Then, a 3-dimensional and 2-dimensional mapping of the magnetic field inside a Helmholtz coil was created in order to investigate the presence of a uniform field, running along the axial direction of the Helmholtz coil. Then, using the obtained results confirm that there is a uniform magnetic field formed inside the Helmholtz coils.  
\section{Theory of the Experiment}
%A short presentation of the concepts and formulas related to the experiment.

% How a helmholtz coil produces a uniform field
Recall for a straight current-carrying wire, circular magnetic field lines are generated around the wire in accordance with the curling right-hand rule. The Helmholtz coil contains two regions of circularly wound wires. (TO BE CONTINUED... how things cancel to make uniform)

\section{Equipment Utilized}
%List principal pieces of apparatus used by manufacturer, model and serial number. When it may be important, list principal specifications of certain pieces of equipment (e.g. the focal length of an optical system, etc.)

%Labeled sketch of the experimental setup
\begin{figure}[h]
\centering
% uncomment the line below to add image
% \includegraphics[width=0.6\textwidth]{sketch.png}
\captionof{figure}{Caption goes here}
\label{diagram}
\end{figure}


\section{Procedure}
% Describe in writing and provide a detailed sketch of the equipment used to measure the axial field in the coil.

%How hall effect probe works


%Describe the main steps in the experimental procedures. Be sure to include any precautions. Sufficient details should be given such that another student can follow and do the experiment.
Note that, per suggestion of the laboratory manual, the procedural steps of this experiment have been omitted. The discussion section provides sufficient detail on what actions were taken.

\subsection{Data Analysis}
%Graphs, figures, and tables with captions
%Results with error analysis
%Calculate discrepancies from theory

%Theoretical calculations of axial field strength

% 3D plot (theoretical)

% 2D plot (theoretical)

% 3D plot (exp)

% 2D plot (exp)

% Calculating discrepancies

% 3D plot discrepancies 

% 2D plot discrepancies

\section{Results}
%Discuss results and uncertainties
%Compare results with theory
%Approximations to theory

% Determine span of uniform region with 1percent margin and 5percent margin

% Compare B field in different direction, can we say field is axial
\subsection{Comparing the directions of the Magnetic Field}
When measuring at a probe height of a/2 (16cm), where 'a' is the separation distance between the coils, 
the strength of the magnetic field in the 'z' direction was measured to be -3.13 Gauss. When measuring t
he magnetic field in the 'z' direction at a probe height of 5cm, the magnetic field strength was measure
d to be -3.28 Gauss. These results follow with the theory as it is expected that the magnetic field is p
ropagated in the 'z' direction. The measured magnetic field strength for the $\rho$ direction was -0.46 
and -0.05 Gauss for a probe height of 16cm and 5cm respectfully. The measured magnetic field strength fo
r the $\phi$ direction was -0.51 and -0.31 Gauss for a probe height of 16cm and 5cm respectfully. For a 
probe height of 16cm the percentage for the magnitude of the magnetic field that is measured to be in th
e $\rho$ direction is 14$\%$ while the percentage for the magnitude of the magnetic field that is measur
ed to be in the $\phi$ direction is 16$\%$. For a probe height of 5cm the percentage for the magnitude o
f the magnetic field that is measured to be in the $\rho$ direction is 1$\%$ while the percentage for th
e magnitude of the magnetic field that is measured to be in the $\phi$ direction is 9$\%$.The magnetic f
ield produced by the Helmholtz coils should be directed along the 'z' axis. These small measuredvalues f
ollow the aforementioned theoryand we can determine that the magnetic field produced by the Helmholtz c
oil is indeed axial. Furthermore, we can determine thatthe magnetic field is axial along the 'z' direct
ion.
\section{Conclusion}
%Brief summary, discussion of results and theory

\end{document}
