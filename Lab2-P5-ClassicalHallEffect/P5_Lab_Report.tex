\documentclass[a4paper]{article}

% Includes packages relevant to Senior Lab

% character set specifications
\usepackage[english]{babel}
\usepackage[utf8]{inputenc}

% extra unicode characters
\DeclareUnicodeCharacter{3BC}{\(\mu\)}
\DeclareUnicodeCharacter{3C1}{\(\rho\)}
\DeclareUnicodeCharacter{2080}{\(_0\)}
\DeclareUnicodeCharacter{2081}{\(_1\)}
\DeclareUnicodeCharacter{2082}{\(_2\)}

% SI Units
\usepackage{siunitx}

% extra SI units
\DeclareSIUnit\gauss{G}

% enable scientific notation
\sisetup{scientific-notation = engineering, exponent-to-prefix}

% draw pretty lines
\usepackage{tikz}
\usetikzlibrary{datavisualization}
\usepackage{circuitikz}

% manual tabbing
\setlength{\parindent}{0pt}
\def\qq{\qquad}

% include graphics
\usepackage{graphicx}

% increased control over figure placement
\usepackage{float}

% box answers
\usepackage{tcolorbox}

% enable multiple section levels
\usepackage{titlesec}

% define `\subsubsubsection` command
\titleclass{\subsubsubsection}{straight}[\subsection]
\newcounter{subsubsubsection}[subsubsection]
\renewcommand\thesubsubsubsection{\thesubsubsection.\arabic{subsubsubsection}}
\titleformat{\subsubsubsection}
        {\normalfont\normalsize\bfseries}{\thesubsubsubsection}{1em}{}
\titlespacing*{\subsubsubsection}
{0pt}{3.25ex plus 1ex minus .2ex}{1.5ex plus .2ex}
\setcounter{secnumdepth}{4}

% get align environment (among other things)
\usepackage{amsmath}

% bold in math mode
\usepackage{bm}

% get \mathbb (among other things)
\usepackage{amssymb}

\usepackage{array}

% plotting
\usepackage{pgfplots}

% enable external references
\usepackage{hyperref}

% include code
\usepackage{minted}
\setminted{linenos, frame=lines, texcomments}


\usepackage{caption}
\renewcommand{\thetable}{\arabic{section}.\arabic{table}}

\title{PHY 4210-01 Senior Lab \\Lab M-1: Magnetic Field Mapping}

\author{Sarah Arends \\
        Jacquelyne Miksanek \\
        Ryan Wojtyla \\ \\
        Instructor: Jerry Collins II}

\date{\today}

\begin{document}
\maketitle

\begin{abstract}
%physics of experiment
%apparatus used
%what was measured
%Results
\qq words words words abstract
\end{abstract}

\newpage

\tableofcontents

\newpage

\section{Objective of the Experiment}
%A brief statement on the main purpose of the experiment
\qq words words words objective

\section{Theory of the Experiment}
%A short presentation of the concepts and formulas related to the experiment.

\section{Equipment Utilized}
%List principal pieces of apparatus used by manufacturer, model and serial number. When it may be important, list principal specifications of certain pieces of equipment (e.g. the focal length of an optical system, etc.)
\begin{itemize}
\item Hall probe
\item Powersource
\item Multimeter
\end{itemize}

\section{Procedure}

% Describe the main steps in the experimental procedures. Be sure to include any
% precautions. Sufficient details should be given such that another student can follow and do the experiment.

\section{Data Analysis}

%Graphs, figures, and tables with captions
%Results with error analysis
%Calculate discrepancies from theory


\section{Hall Effect in N-Germanium}

\subsection{Task 1}

\subsubsection{Data Analysis and Results}
%Graphs, figures, and tables with captions
%Results with error analysis
%Calculate discrepancies from theory
%Discuss results and uncertainties
%Compare results with theory

\subsubsection{Conclusion}
%Brief summary, discussion of results and theory

\subsection{Task 2}

\subsubsection{Data Analysis and Results}
%Graphs, figures, and tables with captions
%Results with error analysis
%Calculate discrepancies from theory
%Discuss results and uncertainties
%Compare results with theory

\subsubsection{Conclusion}
%Brief summary, discussion of results and theory

\subsection{Task 3}

\subsubsection{Data Analysis and Results}
%Graphs, figures, and tables with captions
%Results with error analysis
%Calculate discrepancies from theory
%Discuss results and uncertainties
%Compare results with theory

\subsubsection{Conclusion}
%Brief summary, discussion of results and theory

\subsection{Task 4}

\subsubsection{Data Analysis and Results}
%Graphs, figures, and tables with captions
%Results with error analysis
%Calculate discrepancies from theory
%Discuss results and uncertainties
%Compare results with theory

\subsubsection{Conclusion}
%Brief summary, discussion of results and theory

\subsection{Task 5}

\subsubsection{Data Analysis and Results}
%Graphs, figures, and tables with captions
%Results with error analysis
%Calculate discrepancies from theory
%Discuss results and uncertainties
%Compare results with theory

\subsubsection{Conclusion}
%Brief summary, discussion of results and theory

\newpage

\section{Hall Effect in Pure Germanium}

\subsection{Determining Band Gap in Pure Ge}

\subsubsection{Data Analysis and Results}
%Graphs, figures, and tables with captions
%Results with error analysis
%Calculate discrepancies from theory
%Discuss results and uncertainties
%Compare results with theory

\subsubsection{Conclusion}
%Brief summary, discussion of results and theory

\section{Hall Effect in Pure Zinc}

\subsection{Determining Hall Constant in Pure Zn}

\subsubsection{Data Analysis and Results}
%Graphs, figures, and tables with captions
%Results with error analysis
%Calculate discrepancies from theory
%Discuss results and uncertainties
%Compare results with theory
In order to determine the Hall constant $\text{R}_\text{H}$, one can analyze the dependence of the Hall voltage on the applied field. This was done with a constant DC current applied across the Zn sample, with an associated random error due to the limited precision of the power supply. The error in the slope is obtained through linear regression, and the thickness prescribed by the sample specifications is assumed to have no error.

\begin{center}
\begin{tabular}{|c|c|}
\hline
Slope, $b$ $\big[  \frac{\Omega \text{cm}}{\text{G}} \big] $ & $4.11 \times 10^{-13} \pm ERROR$ \topVspace \bottomVspace \\
\hline
Thickness, $d$ [m] & $2.5 \times 10^{-5} \pm 0$ \topVspace \bottomVspace \\
\hline
Sample Current, I [A] & $13.5 \pm 0.1$ \topVspace \bottomVspace \\
\hline
\end{tabular}
\label{table:zinc_RH}
\captionof{table}{Measurements and calculations used to determine the experimental Hall constant of pure Zinc}
\end{center}

The Hall constant of Zinc is calculated as follows.
\begin{align*}
R_H &= \big( \frac{\mu_H}{B} \big) \, \frac{d}{I} \\
    &= (b) \, \frac{d}{I} \\
    &= (4.11 \times 10^{-13}) \, \frac{2.5 \times 10^{-5}}{13.5} \\
    &= 4.11 \times 10^{-11} \; \text{Vm/TA} \\
    &\equiv 4.11 \times 10^{-13} \; \Omega \text{cm/G} \\
\end{align*}

A theoretical value for the Hall constant of Zinc, given by the third edition of the AIP handbook, is $R_H = 3.30 \times 10^{-13} \; \Omega \text{cm/G}$.

% Propagate error in RH experimental

% Compare theoretical and experimental


\section{Conclusion}
%Brief summary, discussion of results and theory

\section{Appendices}

\subsection{Appendix A: Data}

\subsection{Appendix B: Source Code}

\end{document}
