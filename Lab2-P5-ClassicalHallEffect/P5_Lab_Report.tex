\documentclass[a4paper]{article}

% Includes packages relevant to Senior Lab

% character set specifications
\usepackage[english]{babel}
\usepackage[utf8]{inputenc}

% extra unicode characters
\DeclareUnicodeCharacter{3BC}{\(\mu\)}
\DeclareUnicodeCharacter{3C1}{\(\rho\)}
\DeclareUnicodeCharacter{2080}{\(_0\)}
\DeclareUnicodeCharacter{2081}{\(_1\)}
\DeclareUnicodeCharacter{2082}{\(_2\)}

% SI Units
\usepackage{siunitx}

% extra SI units
\DeclareSIUnit\gauss{G}

% enable scientific notation
\sisetup{scientific-notation = engineering, exponent-to-prefix}

% draw pretty lines
\usepackage{tikz}
\usetikzlibrary{datavisualization}
\usepackage{circuitikz}

% manual tabbing
\setlength{\parindent}{0pt}
\def\qq{\qquad}

% include graphics
\usepackage{graphicx}

% increased control over figure placement
\usepackage{float}

% box answers
\usepackage{tcolorbox}

% enable multiple section levels
\usepackage{titlesec}

% define `\subsubsubsection` command
\titleclass{\subsubsubsection}{straight}[\subsection]
\newcounter{subsubsubsection}[subsubsection]
\renewcommand\thesubsubsubsection{\thesubsubsection.\arabic{subsubsubsection}}
\titleformat{\subsubsubsection}
        {\normalfont\normalsize\bfseries}{\thesubsubsubsection}{1em}{}
\titlespacing*{\subsubsubsection}
{0pt}{3.25ex plus 1ex minus .2ex}{1.5ex plus .2ex}
\setcounter{secnumdepth}{4}

% get align environment (among other things)
\usepackage{amsmath}

% bold in math mode
\usepackage{bm}

% get \mathbb (among other things)
\usepackage{amssymb}

\usepackage{array}

% plotting
\usepackage{pgfplots}

% enable external references
\usepackage{hyperref}

% include code
\usepackage{minted}
\setminted{linenos, frame=lines, texcomments}


\usepackage{caption}
\renewcommand{\thetable}{\arabic{section}.\arabic{table}}

\title{PHY 4210-01 Senior Lab \\Lab P-5: Hall Effect in Semiconductors}

\author{Sarah Arends \\
        Jacquelyne Miksanek \\
        Ryan Wojtyla \\ \\
        Instructor: Jerry Collins II}

\date{February 21, 2019}
\begin{document}
\maketitle

\begin{abstract}
%physics of experiment
%apparatus used
%what was measured
%Results
\qq words words words abstract
\end{abstract}

\newpage

\setcounter{tocdepth}{2}
\tableofcontents

\newpage

\section{Theory of the Experiment}
%A short presentation of the concepts and formulas related to the experiment.

\section{Hall Effect in P-Germanium}

\subsection{Task 1}

\qq The Hall voltage's dependence on the current through the sample
was determined while holding the magnetic field and temperature constant. Please note that the manual called for the magnetic field to be held at 250 mT, 
but this corresponds to a current that exceeds the maximum allowed by 
the coils.
 The field was instead kept at 144 mT, which corresponds to a current of 4A.

\subsubsection{Data Analysis and Results}
%Graphs, figures, and tables with captions
%Results with error analysis
%Calculate discrepancies from theory
%Discuss results and uncertainties

% Discuss results
There is a linear relationship between the Hall Voltage and the sample 
current, which is shown by the following equation, where $\alpha$ is
the proportionality constant:
\begin{align*}
U_H = \alpha * I
\end{align*}
The data was plotted in figure \ref{task21plot}. Through the use of linear regression, the proportionality constant was determined to be $0.943 \pm ERROR$ V/A. 

% Experimental data 
\begin{figure}[H]
\centering
% uncomment the line below to add image
%\includegraphics[width=0.5\textwidth]{FILENAME.png}
\captionof{figure}{Hall voltage as a function of sample current 
in p-type Germanium. The linear fit has an $R^2$ value of 1.}
\label{task21plot}
\end{figure}

% Calculate theoretical proportionality constant
From the equation defining the Hall Voltage, one can derive an expression for $V_{Hall}/I$, which represents the slope or proportionality constant $\alpha$. The expression is used to obtain a value for $\alpha_{theo}$ below, where $B$ is the applied magnetic field, $n$ is the density of charge carriers, $e$ is the elementary charge, and $t$ is the sample thickness.
\begin{align*}
\alpha_{theo} &= \frac{B}{net} \\
			  &= CALCULATION \\
\end{align*}

\subsubsection{Conclusion}
%Brief summary, discussion of results and theory
The experimental constant of proportionality $\alpha = 0.943 $ V/A is within NUMBER $\sigma$ of the theoretical value $\alpha_{theo} = COPYFROMABOVE$ V/A.

\subsection{Task 2}

\qq The sample volatge as a function of the posistive magnetic field
induction was determined. The control current was held at a constant
30 mA. The resistance was computed from the sample voltage. and expressed as a change in resistance relative to the resistance with no field.

\subsubsection{Data Analysis and Results}
%Graphs, figures, and tables with captions
%Results with error analysis
%Calculate discrepancies from theory
%Discuss results and uncertainties
\qq Instead of directly plotting the resistance against the field strength. A "normalized difference" was computed by dividing the total 
change in resistance by the resistance with no applied field, i.e. $\frac{R_m - R_0}{R_0}$, where $R_m$ represents the measured resistance and $R_0$ represent the resistance at $B=0$. The change in resistance associated with a changing magnetic field implies a change in the mean free path of the charge carriers (holes). The resultant graph in figure \ref{task22plot} shows a quadratic change in this expression as a function of the increasing induced magnetic field
strength, which is the expected relationship between the given quantities.

% Experimental data 
\begin{figure}[H]
\centering
% uncomment the line below to add image
%\includegraphics[width=0.5\textwidth]{FILENAME.png}
\captionof{figure}{Normalized change in resistance against applied field strength. The determined $R^2$ value was 0.969}
\label{task22plot}
\end{figure}

\subsubsection{Conclusion}
%Brief summary, discussion of results and theory

\subsection{Task 3}

\qq A constant current of 30 mA is applied and the sample voltage is
measured as a function of temperature. The magnetic field
remains off for this task. The manual states that the maximum
applied temperature is $140^o$ C. In order to avoid damaging the sample, the temperature was instead limited to $110^o$ C. Note that the sample 
was heated to its maximum temperature, and then the sample voltage was taken as the temperature cooled.

\subsubsection{Data Analysis and Results}
%Graphs, figures, and tables with captions
%Results with error analysis
%Calculate discrepancies from theory
%Discuss results and uncertainties
\qq Sample voltage and temperature were measured, with no external field applied. The reciprocal of the voltage was plotted against the 
reciprocal of the temperature. The data, shown in figure \ref{task23plot}, can be represented by a quadratic function, with a minimum occurring around 3 $K^{-1}$.

% Experimental data 
\begin{figure}[H]
\centering
% uncomment the line below to add image
%\includegraphics[width=0.5\textwidth]{FILENAME.png}
\captionof{figure}{Inverse voltage is plotted against inverse temperature.}
\label{task23plot}
\end{figure}
\subsubsection{Conclusion}
%Brief summary, discussion of results and theory

\subsection{Task 4}

\qq The Hall voltage was measured as a function of the magnetic
induction. This time the temperature is held constant at room
temperature, and the current is held constant at 30 mA. The manuel
states that the magnetic field should start at a strength of 300 mT,
however, we started our magnetic field at 290 mT. This was done in
order to avoid overheating the coils and causing damage to the
equipment. Note that the highest current applied to the coils was 5.98
Amperes.

\subsubsection{Data Analysis and Results}
%Graphs, figures, and tables with captions
%Results with error analysis
%Calculate discrepancies from theory
%Discuss results and uncertainties
\qq The Hall voltage was plotted against the magnetic field strength
with both positive and negative polarities. The resultant graph was
linear with a $R^2$ value of 0.999 this value is representative of a
well fit data line.
%Compare results with theory

\subsubsection{Conclusion}
%Brief summary, discussion of results and theory

\subsection{Task 5}

\qq The relationship between the Hall voltage and the temperature is
determined, this time using an induced and constant magnetic
field. The manual calls for the magnetic field to be held constant at
300 mT. However, in order to avoid overheating the coils, we kept the
magnetic field constant at 145 mT. The current was kept steady at 30
mA for the entire task. The manual also states that the temperature
should reach $140^o$ K, however our highest temperature was $90^o$K, this
was done to prevent the overheating of the sample.

\subsubsection{Data Analysis and Results}
%Graphs, figures, and tables with captions
%Results with error analysis
%Calculate discrepancies from theory
%Discuss results and uncertainties
\qq The graph demonstrates a decrease in the Hall voltage as the
temperature increases due to an increase in the number of charge
carries which causes the drift velocity of the charge to decrease.
%Compare results with theory

\subsubsection{Conclusion}
%Brief summary, discussion of results and theory

\section{Hall Effect in N-Germanium}

\subsection{Task 1}

\qq The Hall voltage dependenceon the current was determined while
holding the magnetic field and the temperature at a constant. Please
note that the manual called for the constant magnetic field to be held
at 250 mT, we kept our constant magnetic field at 144 mT in order to
avoid reaching the maximum amperage on the coils. The corresponding
coil current was 4 A.

\subsection{Task 2}

\subsubsection{Data Analysis and Results}
%Graphs, figures, and tables with captions
%Results with error analysis
%Calculate discrepancies from theory
%Discuss results and uncertainties
%Compare results with theory

\subsubsection{Conclusion}
%Brief summary, discussion of results and theory

\subsection{Task 3}

\subsubsection{Data Analysis and Results}
%Graphs, figures, and tables with captions
%Results with error analysis
%Calculate discrepancies from theory
%Discuss results and uncertainties
%Compare results with theory

\subsubsection{Conclusion}
%Brief summary, discussion of results and theory

\subsection{Task 4}

\subsubsection{Data Analysis and Results}
%Graphs, figures, and tables with captions
%Results with error analysis
%Calculate discrepancies from theory
%Discuss results and uncertainties
%Compare results with theory

\subsubsection{Conclusion}
%Brief summary, discussion of results and theory

\subsection{Task 5}

\subsubsection{Data Analysis and Results}
%Graphs, figures, and tables with captions
%Results with error analysis
%Calculate discrepancies from theory
%Discuss results and uncertainties
%Compare results with theory

\subsubsection{Conclusion}
%Brief summary, discussion of results and theory

\newpage

\section{Hall Effect in Pure Germanium}

\subsection{Determining Band Gap in Pure Ge}

\subsubsection{Data Analysis and Results}
%Graphs, figures, and tables with captions
%Results with error analysis
%Calculate discrepancies from theory
%Discuss results and uncertainties
%Compare results with theory

\subsubsection{Conclusion}
%Brief summary, discussion of results and theory

\section{Hall Effect in Pure Zinc}

\subsection{Determining Hall Constant in Pure Zn}

\subsubsection{Data Analysis and Results}
%Graphs, figures, and tables with captions
%Results with error analysis
%Calculate discrepancies from theory
%Discuss results and uncertainties
%Compare results with theory
\qq In order to determine the Hall constant $\text{R}_\text{H}$, one can
analyze the dependence of the Hall voltage on the applied field. This
was done with a constant DC current applied across the Zn sample, with
an associated random error due to the limited precision of the power
supply. The error in the slope is obtained through linear regression,
and the thickness prescribed by the sample specifications is assumed
to have no error.

\begin{center}
\begin{tabular}{|c|c|}
\hline
Slope, $b$ $\big[  \frac{\Omega \text{cm}}{\text{G}} \big] $ & $4.11 \times 10^{-13} \pm ERROR$ \topVspace \bottomVspace \\
\hline
Thickness, $d$ [m] & $2.5 \times 10^{-5} \pm 0$ \topVspace \bottomVspace \\
\hline
Sample Current, I [A] & $13.5 \pm 0.1$ \topVspace \bottomVspace \\
\hline
\end{tabular}
\label{table:zinc_RH}
\captionof{table}{Measurements and calculations used to determine the
  experimental Hall constant of pure Zinc}
\end{center}

% Finding experimental hall constant
The Hall constant of Zinc is calculated as follows.
\begin{align*}
R_{H_{exp}} &= \big( \frac{\mu_H}{B} \big) \, \frac{d}{I} \\
    &= (b) \, \frac{d}{I} \\
    &= (4.11 \times 10^{-13}) \, \frac{2.5 \times 10^{-5}}{13.5} \\
    &= 4.11 \times 10^{-11} \; \text{Vm/TA} \\
    &\equiv 4.11 \times 10^{-13} \; \Omega \text{cm/G} \\
\end{align*}

% Finding experimental aerror
The uncertainty in the experimental Hall constant is propagated as follows.
\begin{align*}
\delta R_{H_{exp}} &= \text{PROPAGATED ERROR IN bd/I} \\
\end{align*}

% Getting theoretical hall constant and error
A theoretical value for the Hall constant of Zinc, given by the third
edition of the AIP handbook, is $R_H = 3.30 \times 10^{-13} \; \Omega 
\text{cm/G}$. This theoretical Hall constant has no given associated error, so the associated discrepancy, $\Delta$, will simply equal that of the experimental Hall constant. The following computations are thus performed.

% Finding discrepancy
\begin{align*}
\Delta &= | R_{H_{exp}} - R_{H_{theo}} | \\
	   &= | (4.11 \times 10^{-13}) - (3.30 \times 10^{-13}) | \\
\end{align*}

% Finding error in discrepancy
\begin{align*}
\delta_{\Delta} &= \delta R_{H_{exp}} \\
				&= \text{COPY VALUE FROM ABOVE} \\
\end{align*}

\subsubsection{Conclusion}
%Brief summary, discussion of results and theory
It is evident that the discrepancy is within SOMENUMBER $\sigma$, so the experimental Hall constant $R_{H_{exp}} = 4.11 \times 10^{-13} \Omega \text{cm/G}$ (AGREES / DOES NOT AGREE) with the reported theoretical value $R_{H_{theo}} = 3.30 \times 10^{-13} \Omega \text{cm/G}$.

\section{Appendices}

\subsection{Appendix A: Data}

\subsection{Appendix B: Source Code}

\end{document}
