\documentclass[a4paper]{article}

% Includes packages relevant to Senior Lab

% character set specifications
\usepackage[english]{babel}
\usepackage[utf8]{inputenc}

% increased vertical spacing for tables
\newcommand\topVspace{\rule{0pt}{2.6ex}}      
\newcommand\bottomVspace{\rule[-1.2ex]{0pt}{0pt}} 

% extra unicode characters
\DeclareUnicodeCharacter{3BC}{\(\mu\)}
\DeclareUnicodeCharacter{3C1}{\(\rho\)}
\DeclareUnicodeCharacter{2080}{\(_0\)}
\DeclareUnicodeCharacter{2081}{\(_1\)}
\DeclareUnicodeCharacter{2082}{\(_2\)}
\DeclareUnicodeCharacter{3B5}{\(\epsilon\)}
\DeclareUnicodeCharacter{3B1}{\(\alpha\)}

% SI Units
\usepackage{siunitx}

% extra SI units
\DeclareSIUnit\gauss{G}

% enable scientific notation
\sisetup{scientific-notation = engineering, exponent-to-prefix}

% draw pretty lines
\usepackage{tikz}
\usetikzlibrary{datavisualization}
\usepackage{circuitikz}

% manual tabbing
\setlength{\parindent}{0pt}
\def\qq{\qquad}

% include graphics
\usepackage{graphicx}

% increased control over figure placement
\usepackage{float}

% box answers
\usepackage{tcolorbox}

% enable multiple section levels
\usepackage{titlesec}

% define `\subsubsubsection` command
\titleclass{\subsubsubsection}{straight}[\subsection]
\newcounter{subsubsubsection}[subsubsection]
\renewcommand\thesubsubsubsection{\thesubsubsection.\arabic{subsubsubsection}}
\titleformat{\subsubsubsection}
        {\normalfont\normalsize\bfseries}{\thesubsubsubsection}{1em}{}
\titlespacing*{\subsubsubsection}
{0pt}{3.25ex plus 1ex minus .2ex}{1.5ex plus .2ex}
\setcounter{secnumdepth}{4}

% get align environment (among other things)
\usepackage{amsmath}

% bold in math mode
\usepackage{bm}

% get \mathbb (among other things)
\usepackage{amssymb}

\usepackage{array}

% plotting
\usepackage{pgfplots}

% enable external references
\usepackage{hyperref}

% include code
\usepackage[cache=false]{minted}
\setminted{linenos, frame=lines, texcomments}

% adjust margins of individual pages (for shoving figures into place)
\usepackage{changepage}

% rotate figures
\usepackage{rotating}


\usepackage{caption}
\renewcommand{\thetable}{\arabic{section}.\arabic{table}}
\newcommand\T{\rule{0pt}{2.6ex}}       % Top strut
\newcommand\B{\rule[-1.2ex]{0pt}{0pt}} % Bottom stru

\title{PHY 4210-01 Senior Lab \\Lab M-1: Magnetic Field Mapping}

\author{Sarah Arends \\ 
        Jacquelyne Miksanek \\
        Ryan Wojtyla \\ \\
        Instructor: Jerry Collins II}

\date{February 21, 2019}
\begin{document}
\maketitle 

\begin{abstract}
%physics of experiment
%apparatus used
%what was measured
%Results
\qq words words words abstract
\end{abstract}

\newpage

\tableofcontents

\newpage

\section{Objective of the Experiment}
%A brief statement on the main purpose of the experiment
\qq words words words objective

\section{Theory of the Experiment}
%A short presentation of the concepts and formulas related to the experiment.


\section{Hall Effect in P-Germanium}

\subsection{Task 1}

\qq The Hall voltage dependence on the current was determined while
holding the magnetic field and the temperature at a constant. Please
note that the manual called for the constant magnetic field to be held
at 250 mT, we kept our constant magnetic field at 144 mT in order to
avoid reaching the maximum amperage on the coils. The corresponding
coil current was 4 A.

\subsubsection{Data Analysis and Results}
%Graphs, figures, and tables with captions
%Results with error analysis
%Calculate discrepancies from theory
%Discuss results and uncertainties
There is a linear relationship between the Hall Voltage and the
current, which is shown by the following equation, where $\alpha$ is
the proportionality constant:
\begin{align*}
U_H = \alpha * I
\end{align*}
The obtained proportionality constant was determined to be 0.943 V/A,
and the obtained R^2 value was 1.
%Compare results with theory

\subsubsection{Conclusion}
%Brief summary, discussion of results and theory

\subsection{Task 2}

\qq The sample volatge as a function of the posistive magnetic field
induction was determined. The control current was held at a constant
30 mA. The change in resistance is to be compared to an increasing
field strength.

\subsubsection{Data Analysis and Results}
%Graphs, figures, and tables with captions
%Results with error analysis
%Calculate discrepancies from theory
%Discuss results and uncertainties
\qq The resultant graph shows a non-linear and quadratic change in the
resistance as a function of the increasing induced magnetic field
strength. The determined R^2 value was 0.978 representing a well-fit
data line.
%Compare results with theory

\subsubsection{Conclusion}
%Brief summary, discussion of results and theory

\subsection{Task 3}

\qq A constant current of 30 mA is applied and the sample voltage is
measured as a function with its temperature. The magnetic field
remains off for this task. The manuel states that the maximum
temperature to be measured is 140^oC. In order to not overload the
sample the maximum temperature we obtained was 90^oC. Note that the
maximum temperature was reached then the sample voltage was taken as
the temperature cooled.

\subsubsection{Data Analysis and Results}
%Graphs, figures, and tables with captions
%Results with error analysis
%Calculate discrepancies from theory
%Discuss results and uncertainties
\qq The reciprocal of the voltage was plotted against the reciprocal
of the temperature. The graph has a turn around point at about 3
inverse temperature where it goes from decreasing to increasing.
%Compare results with theory

\subsubsection{Conclusion}
%Brief summary, discussion of results and theory


\section{Hall Effect in N-Germanium}

\subsection{Task 1}

\qq The Hall voltage dependenceon the current was determined while
holding the magnetic field and the temperature at a constant. Please
note that the manual called for the constant magnetic field to be held
at 250 mT, we kept our constant magnetic field at 144 mT in order to
avoid reaching the maximum amperage on the coils. The corresponding
coil current was 4 A.

\subsubsection{Data Analysis and Results}
%Graphs, figures, and tables with captions
%Results with error analysis
%Calculate discrepancies from theory
%Discuss results and uncertainties
%Compare results with theory

\subsubsection{Conclusion}
%Brief summary, discussion of results and theory

\subsection{Task 2}

\subsubsection{Data Analysis and Results}
%Graphs, figures, and tables with captions
%Results with error analysis
%Calculate discrepancies from theory
%Discuss results and uncertainties
%Compare results with theory

\subsubsection{Conclusion}
%Brief summary, discussion of results and theory

\subsection{Task 3}

\subsubsection{Data Analysis and Results}
%Graphs, figures, and tables with captions
%Results with error analysis
%Calculate discrepancies from theory
%Discuss results and uncertainties
%Compare results with theory

\subsubsection{Conclusion}
%Brief summary, discussion of results and theory

\subsection{Task 4}

\subsubsection{Data Analysis and Results}
%Graphs, figures, and tables with captions
%Results with error analysis
%Calculate discrepancies from theory
%Discuss results and uncertainties
%Compare results with theory

\subsubsection{Conclusion}
%Brief summary, discussion of results and theory

\subsection{Task 5}

\subsubsection{Data Analysis and Results}
%Graphs, figures, and tables with captions
%Results with error analysis
%Calculate discrepancies from theory
%Discuss results and uncertainties
%Compare results with theory

\subsubsection{Conclusion}
%Brief summary, discussion of results and theory

\section{Hall Effect in Pure Germanium}

\subsection{Determining Band Gap in Pure Ge}

\subsubsection{Data Analysis and Results}
%Graphs, figures, and tables with captions
%Results with error analysis
%Calculate discrepancies from theory
%Discuss results and uncertainties
%Compare results with theory

\subsubsection{Conclusion}
%Brief summary, discussion of results and theory

\section{Hall Effect in Pure Zinc}

\subsection{Determining Hall Constant in Pure Zn}

\subsubsection{Data Analysis and Results}
%Graphs, figures, and tables with captions
%Results with error analysis
%Calculate discrepancies from theory

\section{Appendices}

\subsection{Appendix A: Data}

\subsection{Appendix B: Source Code}

\end{document}
