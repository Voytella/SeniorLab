\documentclass[a4paper]{article}

% Includes packages relevant to Senior Lab

% character set specifications
\usepackage[english]{babel}
\usepackage[utf8]{inputenc}

% extra unicode characters
\DeclareUnicodeCharacter{3BC}{\(\mu\)}
\DeclareUnicodeCharacter{3C1}{\(\rho\)}
\DeclareUnicodeCharacter{2080}{\(_0\)}
\DeclareUnicodeCharacter{2081}{\(_1\)}
\DeclareUnicodeCharacter{2082}{\(_2\)}

% SI Units
\usepackage{siunitx}

% extra SI units
\DeclareSIUnit\gauss{G}

% enable scientific notation
\sisetup{scientific-notation = engineering, exponent-to-prefix}

% draw pretty lines
\usepackage{tikz}
\usetikzlibrary{datavisualization}
\usepackage{circuitikz}

% manual tabbing
\setlength{\parindent}{0pt}
\def\qq{\qquad}

% include graphics
\usepackage{graphicx}

% increased control over figure placement
\usepackage{float}

% box answers
\usepackage{tcolorbox}

% enable multiple section levels
\usepackage{titlesec}

% define `\subsubsubsection` command
\titleclass{\subsubsubsection}{straight}[\subsection]
\newcounter{subsubsubsection}[subsubsection]
\renewcommand\thesubsubsubsection{\thesubsubsection.\arabic{subsubsubsection}}
\titleformat{\subsubsubsection}
        {\normalfont\normalsize\bfseries}{\thesubsubsubsection}{1em}{}
\titlespacing*{\subsubsubsection}
{0pt}{3.25ex plus 1ex minus .2ex}{1.5ex plus .2ex}
\setcounter{secnumdepth}{4}

% get align environment (among other things)
\usepackage{amsmath}

% bold in math mode
\usepackage{bm}

% get \mathbb (among other things)
\usepackage{amssymb}

\usepackage{array}

% plotting
\usepackage{pgfplots}

% enable external references
\usepackage{hyperref}

% include code
\usepackage{minted}
\setminted{linenos, frame=lines, texcomments}


\usepackage{caption}
\renewcommand{\thetable}{\arabic{section}.\arabic{table}}
\newcommand\T{\rule{0pt}{2.6ex}}       % Top strut
\newcommand\B{\rule[-1.2ex]{0pt}{0pt}} % Bottom strut

\title{PHY 4210-01 Senior Lab \\Lab P2: Electron Spin Resonance}

\author{Sarah Arends \\ 
        Jacquelyne Miksanek \\
        Ryan Wojtyla \\ \\
        Instructor: Jerry Collins}

\date{March 28, 2019}

\begin{document}
\maketitle 

\begin{abstract}
%physics of experiment
%apparatus used
%what was measured
%Results

\qq The Lande factor, $g_s$, the gyromagnetic ratio of spin for the
electron was determined through the use of electron spin resonance and
Helmholtz coils. The g-factor of a diphenyl-picryl-hydrazyl (DPPH) sample was
obtained following the measurement of the frequency dependence of the
resonance field. The line width of the resonance signal was then calculated.

\end{abstract}

\newpage

\tableofcontents

\newpage

\section{Data Analysis}
%Graphs, figures, and tables with captions
%Results with error analysis
%Calculate discrepancies from theory
\subsection{Frequency Dependence of Resonance Field}

\qq Voltage was compared to frequency to obtain a graphical
relationship of the frequency dependence of the resonance field. The
amplitude voltage was obtained by measuring the peak-to-peak voltage
from the oscilloscope and dividing it in half. The peak of
\ref{FrequencyDependece} is the specific resonance frequency for the field.
\begin{figure}[H]
\centering
% uncomment the line below to add image
%\includegraphics[scale=2.0]{freq_depen.png}
\captionof{figure}{Graphic representation of the frequency dependence
  of the resonance field.}
\label{FrequencyDependence}
\end{figure}

\subsection{Propagration of Uncertianty in the Frequency Dependence of the Resonance Field}

\subsection{Experimental Value of Gyromagnetic Ratio}
\qq The gyromagnetic ratio is calculated using the following equation, where $\nu$ is the frequency, $h$ is Planck's constant, $\mu_B$ is the Bohr magneton, and $B_0$ is the magnetic field strength.
\begin{equation}
\label{eq:exp_gs}
g_s = \frac{h \times \nu}{\mu_B \times B_0}
\end{equation} 

\qq The magnetic field used in calculating equation \ref{eq:exp_gs} must be calculated as well. It is determined from the measured current using equation \ref{eq:exp_B}, where $\mu_0 = 4 \pi \times 10^{-7} \frac{Vs}{Am}$, the number of turns is $n=320$, and the radius of the coils is $r=6.8cm$.
\begin{equation}
\label{eq:exp_B}
B_0 = \mu_0 \left( \frac{4}{5} \right) ^{3/2} \times \frac{n}{r} \times I
\end{equation}

\qq Rather than measuring the current directly, the current is calculated by measuring the voltage drop across a resistor, of which the resistance is also measured. This calculation is shown below in equation.
\begin{equation}
\label{eq:exp_I}
I = \frac{V}{R}
\end{equation}

\qq By substituting equation \ref{eq:exp_I} into \ref{eq:exp_B}, and then substituting equation \ref{eq:exp_B} into equation \ref{eq:exp_gs}, we arrive at an expression for the gyromagnetic ratio in terms of known constants and measured quantities. This final expression is shown in equation \ref{eq:exp_gs_combined}.
\begin{equation}
\label{eq:exp_gs_combined}
g_s = \frac{h \times \nu}{\mu_B \times \left( \mu_0 \left( \frac{4}{5} \right) ^{3/2} \times \frac{n}{r} \times \frac{V}{R} \right) }
\end{equation}

\subsection{Propagating Uncertainty in Gyromagnetic Ratio}
\qq The error in the experimental value of the gyromagnetic ratio is determined by propagating uncertainty in equation \ref{eq:exp_gs_combined}.

\begin{align*}
\delta g_s &= \\
\end{align*}

\subsection{Determining Line Width of Resonance Signal}
\qq 

\section{Results}
%Discuss results and uncertainties
%Compare results with theory
%Approximations to theory
\qq

\subsection{Discrepancy in Gyromagnetic Ratio}
\qq 

\subsection{Discrepancy in Line Width}
\qq 

\section{Conclusion}
%Brief summary, discussion of results and theory
\qq 

\section{Appendices}

\subsection{Appendix A: Data}

\subsection{Appendix B: Source Code}

\end{document}
