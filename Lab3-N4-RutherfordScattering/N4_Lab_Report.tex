\documentclass[a4paper]{article}

% Includes packages relevant to Senior Lab

% character set specifications
\usepackage[english]{babel}
\usepackage[utf8]{inputenc}

% increased vertical spacing for tables
\newcommand\topVspace{\rule{0pt}{2.6ex}}      
\newcommand\bottomVspace{\rule[-1.2ex]{0pt}{0pt}} 

% extra unicode characters
\DeclareUnicodeCharacter{3BC}{\(\mu\)}
\DeclareUnicodeCharacter{3C1}{\(\rho\)}
\DeclareUnicodeCharacter{2080}{\(_0\)}
\DeclareUnicodeCharacter{2081}{\(_1\)}
\DeclareUnicodeCharacter{2082}{\(_2\)}
\DeclareUnicodeCharacter{3B5}{\(\epsilon\)}
\DeclareUnicodeCharacter{3B1}{\(\alpha\)}

% SI Units
\usepackage{siunitx}

% extra SI units
\DeclareSIUnit\gauss{G}

% enable scientific notation
\sisetup{scientific-notation = engineering, exponent-to-prefix}

% draw pretty lines
\usepackage{tikz}
\usetikzlibrary{datavisualization}
\usepackage{circuitikz}

% manual tabbing
\setlength{\parindent}{0pt}
\def\qq{\qquad}

% include graphics
\usepackage{graphicx}

% increased control over figure placement
\usepackage{float}

% box answers
\usepackage{tcolorbox}

% enable multiple section levels
\usepackage{titlesec}

% define `\subsubsubsection` command
\titleclass{\subsubsubsection}{straight}[\subsection]
\newcounter{subsubsubsection}[subsubsection]
\renewcommand\thesubsubsubsection{\thesubsubsection.\arabic{subsubsubsection}}
\titleformat{\subsubsubsection}
        {\normalfont\normalsize\bfseries}{\thesubsubsubsection}{1em}{}
\titlespacing*{\subsubsubsection}
{0pt}{3.25ex plus 1ex minus .2ex}{1.5ex plus .2ex}
\setcounter{secnumdepth}{4}

% get align environment (among other things)
\usepackage{amsmath}

% bold in math mode
\usepackage{bm}

% get \mathbb (among other things)
\usepackage{amssymb}

\usepackage{array}

% plotting
\usepackage{pgfplots}

% enable external references
\usepackage{hyperref}

% include code
\usepackage[cache=false]{minted}
\setminted{linenos, frame=lines, texcomments}

% adjust margins of individual pages (for shoving figures into place)
\usepackage{changepage}

% rotate figures
\usepackage{rotating}


\usepackage{caption}
\renewcommand{\thetable}{\arabic{section}.\arabic{table}}
\newcommand\T{\rule{0pt}{2.6ex}}       % Top strut
\newcommand\B{\rule[-1.2ex]{0pt}{0pt}} % Bottom strut

\title{PHY 4210-01 Senior Lab \\Lab N4: Rutherford Scattering}

\author{Sarah Arends \\ 
        Jacquelyne Miksanek \\
        Ryan Wojtyla \\ \\
        Instructor: Dr. Marcus Hohlmann}

\date{March 14, 2019}

\begin{document}
\maketitle 

\begin{abstract}
%physics of experiment
%apparatus used
%what was measured
%Results
  \qq 
\end{abstract}

\newpage

\tableofcontents

\newpage

\section{Objective of the Experiment}
%A brief statement on the main purpose of the experiment
\qq During this experiment, the differential cross-section for a scattering process will be determined. Researchers will measure the counting rate for alpha particles scattered by a Gold or Aluminum foil as a function of the angle at which it is scattered. Using this information, one can calculate a counting rate corrected with respect to the scattering distribution. The following Rutherford scattering formula can then be validated:
\begin{equation}
N(\theta) = N_0 \times c_F \times d_f \times 
            \frac {Z^2 \times e^4}
                  {
                  \left( 8\pi \times \epsilon_0\times E_{\alpha} \right) ^2 
                   \times  sin^4 \left( \frac{\theta}{2} \right)
                  }      
\end{equation}

\section{Theory of the Experiment}
%A short presentation of the concepts and formulas related to the experiment. 

% How Rutherford scattering works (Coulomb repulsion etc)
\qq Rutherford scattering describes the process in which charged particles article undergo elastic scattering due to a Coulomb force interaction. When the positively charged
alpha particles approach the positively charged gold nuclei, the like charges cause a repulsive force that deflects the alpha particles at varying angles. This deflection/scattering angle depends on the distance of closest approach between the alpha particles and nuclei, since the Coulomb force is a function of distance. Because the gold atoms consist of mostly empty space, the majority of alpha particles are sufficiently far away from the gold nuclei that they experience minimal Coulomb repulsion and are only scattered at angles of less than one degree. However, there are still some alpha particles that approach the gold nuclei close enough to experience stronger Coulomb repulsion and scattering at greater angles. If one were to observe the number of scattered alpha particles as a function of the angle at which they are scattered, the resultant distribution would show a large rate at small angles, but the distribution would quickly drop off as the angle increases. Note, however, that this distribution does not account for particles that are back-scattered, meaning that they are close enough to the gold nuclei to be deflected backwards towards the alpha source. Accounting for these measurements would show a spike in the rate curve at an angle of 180 degrees, in what is otherwise a monotonically decreasing distribution. 

% Physical meaning and derivation of differential cross section
\qq 

% Distribution of scattering function vs angle shape
\qq 

\section{Equipment Utilized}
%List principal pieces of apparatus used by manufacturer, model and
%serial number. When it may be important, list principal
%specifications of certain pieces of equipment (e.g. the focal length
%of an optical system, etc.)

% Description of set-up in prose
\qq The readout electronics system is set-up in series beginning at
the Rutherford scattering chamber's photodiode which is connected
directly to the preamplifier which connects to the amplifier then the
discriminator followed by the counter. The oscilloscope is essentially
connected in parallel, and was moved around in the set-up to check
signals at all points in the circuit design. The Rutherford scattering
chamber is also connected to the vacuum pump, which is utilized to
place the contents of the chamber under vacuum. Rutherford's chamber
is comprised of several components. The first component is the
americium 241 which is closely followed by the collimating slit and
gold ( or aluminum ) foil, the final component being the
photodiode. The photodiode detects the energy of the alpha particle
and converts it into an electrical signal, which is then transmitted
through the readout system.

% List of specs
\begin{itemize}
\item Rutherford scattering chamber 
\item Aluminum foil in frame:  \\
      Molar mass 27 g/mol \\
      Thickness $1.50 \times 10^{-7}$ m \\
      Density $2.70 \times 10^6$ g/m$^3$ 
\item Gold foil in frame: \\
      Molar mass 197 g/mol \\
      Thickness $2.00 \times 10^{-8}$ m \\
      Density $1.93 \times 10^7$ g/m$^3$ 
\item Vacuum pump, for evacuating scattering chamber
\item Readout modules: \\
      Discriminator, Amplifier, Preamplifier, Counter
\item $ ^{241}$Americium (alpha source)
\item Oscilloscope, for monitoring signals on readout modules
\item Photodiode detector \\
	  Width $2.22 \times 10^{-3}$ m \\
	  Height $4.12 \times 10^{-3}$ m 
\item Collimating slit \\
      Width 0.005 m 
\end{itemize}

%Labeled sketch of the experimental setup
\begin{figure}[H]
\centering
% uncomment the line below to add image
%\includegraphics[width=0.5\textwidth]{figure.png}
\captionof{figure}{Description of schematic here}
\label{name}
\end{figure}

\section{Procedure}

% Describe the main steps in the experimental procedures. Be sure to include any
% precautions. Sufficient details should be given such that another student can
% follow and do the experiment.

% Production in gas chamber
\qq Alpha particles are produced from an $^{241}$Am source mounted inside the vacuum chamber on a rotating arm. Also mounted on this arm, directly in front of the source, is a collimating slit followed by a thin gold foil. With this configuration, one is able to strike the gold foil with a uniform, collimated beam of alpha particles. The chamber must be evacuated using an external vacuum pump, since alpha particles have a very short lifetime in air.

% Changing angle and measuring effect
\qq Alpha particles are scattered by Gold nuclei at varying angles, and the scattered particles are detected using a photodiode. In order to determine the dependence of the scattering rate on the incident angle, the rotating arm was moved through a range of $-30$ degrees to $30$ degrees using a knob on top of the vacuum chamber. Here, an angle of 0 degrees represents the arm oriented along the same line as the photodiode.

% Readout electronics
\qq The photodiode is connected via an external BNC port to a pre-amplifier that shapes the measurement signal. This is then connected to an amplifier, which amplifies the signal in accordance with a prescribed gain. This amplified signal is fed to a discriminator, which sets a minimum/threshold voltage in order to differentiate meaningful signals from electronic noise. Signals that meet this threshold are output in the form of a digital pulse. These pulses are fed to a counter module, which reads out the number of pulses over a given time interval.

% Acquisition time
The count rate decreases dramatically as the scattering angle increases. For this reason, the acquisition time was increased for larger angles in order to obtain sufficient statistics. The experimental scattering rate can be used to calculate the differential cross-section.

\subsection{Procedural Modifications}
\qq The manual dictates that the chamber be evacuated using the vacuum pump before conducting trials. However, it was determined that the seal of the vacuum chamber was not sufficient, and the detected rate of alpha particles would dramatically decrease after a few minutes. Therefore, in order to maintain a true vacuum throughout the course of data taking, the pump was continually operated throughout the experiment.
\qq 

\section{Data Analysis}

\subsection{Data Analysis I: Gold}
%Graphs, figures, and tables with captions
%Results with error analysis
%Calculate discrepancies from theory

% Direct counting rate as a function of angle (gold)
\qq

% Corrected counting rate (wrt. distribution in space)
\qq 

% Experimental results for rutherford formula
\qq 

% Experimental uncertainty (\delta)
\qq 

% Theoretical results for rutherford formula
\qq 

\subsection{Data Analysis II: Aluminum}
%Graphs, figures, and tables with captions
%Results with error analysis
%Calculate discrepancies from theory

% Direct counting rate as a function of angle (gold)
\qq

% Corrected counting rate (wrt. distribution in space)
\qq 

% Experimental results for rutherford formula
\qq 

% Experimental uncertainty (\delta)
\qq 

% Theoretical results for rutherford formula
\qq 

\section{Results}

\subsection{Results I: Gold}
%Discuss results and uncertainties
%Compare results with theory
%Approximations to theory

% Tabulate theo and exp values/uncertainties
\qq 

% Calculate the difference/discrepancy (\Delta)
\qq 

% Calculate the uncertainty in the discrepancy
% Quantify how many sigmas the discrepancy is
\qq

\subsection{Results II: Aluminum}
%Discuss results and uncertainties
%Compare results with theory
%Approximations to theory

% Tabulate theo and exp values/uncertainties
\qq 

% Calculate the difference/discrepancy (\Delta)
\qq 

% Calculate the uncertainty in the discrepancy
% Quantify how many sigmas the discrepancy is
\qq

\section{Conclusion}
%Brief summary, discussion of theory

% Results for gold
% Note the number of sigmas (discrepancy) and what that means statistically
\qq 

% Results for aluminum
% Note the number of sigmas (discrepancy) and what that means statistically
\qq 

\section{Appendices}

\subsection{Appendix A: Data}

\subsection{Appendix B: Source Code}

\end{document}
